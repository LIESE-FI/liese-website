\chapter{Clonar un repositorio remoto}

Clonar un repositorio es la acción de descargar el repositorio de GitHub de otro usuario, sin vincularlo con el nuestro. SOLO se hace una copia de los archivos, pero no existe vinculación entre repositorios. Para clonar un repositorio de otro usuario, se deben seguir los siguientes pasos:

\begin{enumerate}
    \item \textbf{Creamos un nuevo directorio}
    
    Creamos una nueva carpeta, que será nuestro Workspace, de la misma manera en la que lo hacemos cuando queremos crear un repositorio local.
    
    \item \textbf{Inicializamos el repositorio}
    
    En el directorio creado, inicializamos el repositorio con el comando "\textbf{\textit{git init}}", como lo hariamos normalmente

    \item \textbf{Nos dirigimos al repositorio a clonar}

    Nos dirigimos a la pagina donde se encuentre el repositorio a clonar

    \begin{figure}[H]
        \centering
        \includegraphics[scale = 0.7]{Imagenes/9.0.1-1.png}
        \caption{Pagina del repositorio a clonar}
        \label{}
     \end{figure}

    \item \textbf{Copiamos el enlace del repositorio}

    Posteriormente copiamos el enlace de la dirección donde se encuentra el repositorioa copiar. Para esto damos clic a la opción \textbf{\textit{Code}}, seleccionamos \textbf{\textit{HTTPS}}, y posteriormente copiamos el link.

    \begin{figure}[H]
        \centering
        \includegraphics[scale = 0.8]{Imagenes/9.0.1-2.png}
        \caption{Copiamos el enlace del repositorio a clonar}
        \label{}
     \end{figure} 
    
\end{enumerate}
