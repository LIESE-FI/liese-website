\chapter{Creación, configuración y manejo de un repositorio local}

\section{Creación del repositorio local y el espacio de trabajo}

Para crear la carpeta de proyecto, que se usará como repositorio local, se deben seguir los siguientes pasos:

\begin{enumerate}

    \item \textbf{Creación del workspace (espacio de trabajo)}
    
        Creamos una nueva carpeta, y le damos el nombre que se desee. Para fines didacticos, en este ejemplo se dará el nombre de \textit{"Proyecto"}
        
        \begin{figure}[H]
            \centering
            \includegraphics[scale = 0.9]{Imagenes/1creacarp.png}
            \caption{Creación de una nueva carpeta}
            \label{}
        \end{figure}

        Esta carpeta será nuestro \textit{\textbf{workspace (espacio de trabajo)}}.
        
    \item \textbf{Añadir los archivos}
    
    Agregamos los archivos que serán parte del proyecto a nuestro espacio de trabajo (a la carpeta de proyecto).
        
        \begin{figure}[H]
            \centering
            \includegraphics[scale = 1]{Imagenes/2AddArch.png}
            \caption{Archivos en el espacio de trabajo}
            \label{}
        \end{figure}

    \item \textbf{Abrir la terminal de Git Bash}
    
    Ahora abrimos la terminal de Git Bash en nuestro espacio de trabajo; para esto hacemos \textbf{\textit{click derecho}} en cualquier parte de la carpeta del proyecto, y despues seleccionamos la opción \textbf{\textit{Open Git Bash here}}.
        
        \begin{figure}[H]
            \centering
            \includegraphics[scale = 1]{Imagenes/3openbash.png}
            \caption{Abriendo la terminal de Git Bash}
            \label{}
        \end{figure}

        Esto nos abrirá la terminal de Git Bash direccionada a la ruta actual (ubicación del espacio de trabajo).
        
        \begin{figure}[H]
            \centering
            \includegraphics[scale = 1]{Imagenes/3cons.png}
            \caption{Consola de Git Bash}
            \label{}
        \end{figure}

        Otra opción, en vez de usar la terminal de Git Bash, es abrir la \textit{\textbf{terminal de Windows}}. Para esto, nos dirigimos a la parte donde se encuentra la ruta de la carpeta de proyecto, \textbf{borramos la ruta}, y escribimos \textbf{\textit{cmd}}.
            
        \begin{figure}[H]
            \centering
            \includegraphics[scale = 1]{Imagenes/cmd.png}
            \caption{Abriendo la Terminal de Windows}
            \label{}
        \end{figure}
        
        Esto nos abrirá la terminal de Windows direccionada a la ruta actual (ubicación del espacio de trabajo).
        
        \begin{figure}[H]
            \centering
            \includegraphics[scale = 0.9]{Imagenes/CMDwindows.png}
            \caption{Consola de Git Bash}
            \label{}
        \end{figure}

    \item \textbf{Inicializar el repositorio local}
    
    Para Inicializar el repositorio local escrimos el comando \textit{\textbf{"git init"}} en la terminal de Git Bash como se muestra a continuación.
    
        \begin{figure}[H]
            \centering
            \includegraphics[scale = 1]{Imagenes/4-init.png}
            \caption{Inicialización del repositorio local}
            \label{}
        \end{figure}
        
        Esto inicializa el repositorio local en la carpeta del proyecto, creando asi el archivo \textbf{\textit{.git}}, que será el gestor de archivos.

        \begin{figure}[H]
            \centering
            \includegraphics[scale = 1]{Imagenes/4-ArchGit.png}
            \caption{Archivo \textit{.git}}
            \label{}
        \end{figure}

\end{enumerate}

\section{Configuración del repositorio local}

Al repositorio local se le pueden hacer ciertas configuraciónes, tales como:

\begin{itemize}
    \item Configurar nombre de usuario
    \item Asociar un correo electrónico
    \item Asociar un IDE para el manejo de Git
    \item Configurar saltos de linea
    
\end{itemize}

\textbf{\underline{Configuración del nombre de usuario}}

Para configurar el nombre de usuario al proyecto, se escribe el comando \textit{\textbf{git config $--$global user.name}}, seguido del nombre de usuario entre comillas. A continuación se muestra un ejemplo:
        
    \begin{figure}[H]
        \centering
        \includegraphics[scale = 1]{Imagenes/confname.png}
        \caption{Configurar nombre de usuario}
        \label{}
    \end{figure}

\textbf{\underline{Asociar un correo electrónico}}

Para asociar un correo electrónico al proyecto, se escribe el comando \textit{\textbf{git config $--$global user.email}}, seguido del correo electrónico del usuario. A continuación se muestra un ejemplo:
        
    \begin{figure}[H]
        \centering
        \includegraphics[scale = 1]{Imagenes/ConfCorreo.png}
        \caption{Asociar un correo electrónico}
        \label{}
    \end{figure}

\textbf{\underline{Configurar saltos de linea}}

El formato es la fuente de los problemas más sutiles y frustrantes que muchos desarrolladores pueden encontrar en entornos colaborativos, especialmente si son multi-plataforma.

Es muy fácil que algunos parches u otros trabajos recibidos introduzcan sutiles cambios de espaciado, ya sea porque los editores suelen hacerlo inadvertidamente, o (trabajando en entornos multi-plataforma) si los archivos alguna vez tocan un sistema operativo distinto al del usuario sus finales de línea podrían ser reemplazados.

Si se está trabajando con personas que tienen otro sistema operativo, probablemente aparecerán problemas de final de línea en algún momento.

 Esto se debe a que Windows utiliza los codigos de control \textbf{\textit{CRLF}} (retorno-de-carro y salto-de-linea) para marcar los finales de línea de sus archivos. Mientras que Mac y Linux utilizan solamente el codigo de control \textbf{\textit{LF}} (el carácter de salto-de-linea). Esta es una sutil, pero molesta, diferencia cuando se trabaja en entornos multi-plataforma.

Muchos editores en Windows reemplazan silenciosamente los finales de línea de estilo \textbf{\textit{LF}} existentes con \textbf{\textit{CRLF}}, o insertan ambos caracteres de final de línea. Git maneja esto convirtiendo automáticamente los finales de línea \textbf{\textit{CRLF}} en \textbf{\textit{LF}}, y viceversa.

Para que Git pueda convertir los finales de linea \textbf{\textit{CRLF}} en \textbf{\textit{LF}}, y viceversa, debe hacerse uso del comando \textbf{\textit{git config --global core.autocrlf}} de la siguiente manera dependiendo del sistema operativo:

\begin{table}[H]
\centering
\begin{tabular}{|
>{\columncolor[HTML]{EFEFEF}}c |c|}
\hline
\cellcolor[HTML]{96FFFB}{\color[HTML]{FE0000} \textbf{Sistema Operativo}} & \cellcolor[HTML]{96FFFB}{\color[HTML]{FE0000} \textbf{Comando}} \\ \hline
Windows                                                                   & \textit{\textbf{git config --global core.autocrlf true}}        \\ \hline
Linux o Mac                                                               & \textit{\textbf{git config --global core.autocrlf input}}       \\ \hline
\end{tabular}
\end{table}

A continuación, se muestra un ejemplo:
        
    \begin{figure}[H]
        \centering
        \includegraphics[scale = 1]{Imagenes/SaltLin.png}
        \caption{Comando para configurar los saltos de linea}
        \label{}
    \end{figure}

\textbf{\underline{Ver la configuración del repositorio local}}

Para ver la configuración del repositorio local, se escribe el comando \textit{\textbf{git config $--$global -e}}. A continuación, se muestra un ejemplo:
        
    \begin{figure}[H]
        \centering
        \includegraphics[scale = 1]{Imagenes/VerConf.png}
        \caption{comando para ver la configuración del repositorio local}
        \label{}
    \end{figure}

La información del repositorio local se mostrará en pantalla, como se ve a continución:

\section{Manejo del staging area (área de preparación)}


\subsection{Ver el estado de los archivos}

En Git es posible ver el estado de los archivos como lo son los archivos agregados al Staging Area, los archivos confirmados, los archivos en el espacio de trabajo, etc. Para ver el estado de los archivos se utiliza el comando \textbf{\textit{git status}}. Existen varias maneras de ver el estado de los archivos:

\begin{table}[H]
\centering
\begin{tabular}{|
>{\columncolor[HTML]{EFEFEF}}l |
>{\columncolor[HTML]{FFFFFF}}c |}
\hline
\multicolumn{1}{|c|}{\cellcolor[HTML]{96FFFB}{\color[HTML]{FE0000} \textbf{Acción}}}         & \cellcolor[HTML]{96FFFB}{\color[HTML]{FE0000} \textbf{Comando}} \\ \hline
Ver el estado de los archivos                                                                & \textit{\textbf{git status}}                                    \\ \hline
\begin{tabular}[c]{@{}l@{}}Ver el estado de los archivos\\ de manera condensada\end{tabular} & \textit{\textbf{git status -s}}                                 \\ \hline
\end{tabular}
\end{table}

\textbf{\underline{Ejemplo:}}

A continuación se muestra un ejemplo de como ver el estado de los archivos:

    \begin{figure}[H]
        \centering
        \includegraphics[scale = 1]{Imagenes/GitStatus.png}
        \caption{Ejemplo de como ver el estado de los archivos}
        \label{}
    \end{figure}

En el ejemplo podemos ver:

\begin{itemize}
    \item La rama en la que nos encontramos
    \item Si ha habido confirmaciones
    \item El estado de los archivos (rastreados, no rastreados)
    \item Que archivos han sido modificados
    \item Asi como algunas sugerencias
\end{itemize}

\subsection{Preparar y añadir archivos al staging area}

Como se dijo anteriormente, todos los Archivos que sean añadidos a la carpeta de proyecto, se consideran dentro del \textbf{\textit{workspace (espacio de trabajo)}}; Sin embargo, a pesar de que los archivos se encuentren dentro del Workspace, esto no significa que estos archivos esten siendo rastreados por el gestor de archivos.

Para que los archivos dentro del \textbf{\textit{workspace}} sean tomados en cuenta (rastreados) por el gestor de archivos, es necesario añadirlos al \textbf{\textit{staging area (área de preparación)}}, de otra forma el gestor de archivos no los rastreará, y no tomará en cuenta los cambios que se les realicen.

Para añadir archivos del workspace al staging area, se utiliza el comando \textbf{\textit{git add}}. Existen multiples maneras de agregar archivos al Staging Area utilizando el comando \textbf{\textit{git add}}:

\begin{table}[H]
\centering
\begin{tabular}{|
>{\columncolor[HTML]{EFEFEF}}l |
>{\columncolor[HTML]{FFFFFF}}c |
>{\columncolor[HTML]{FFFFFF}}l |}
\hline
\multicolumn{1}{|c|}{\cellcolor[HTML]{96FFFB}{\color[HTML]{FE0000} \textbf{Acción}}}                              & \cellcolor[HTML]{96FFFB}{\color[HTML]{FE0000} \textbf{Comando}}                      & \multicolumn{1}{c|}{\cellcolor[HTML]{96FFFB}{\color[HTML]{FE0000} \textbf{Ejemplo}}} \\ \hline
\begin{tabular}[c]{@{}l@{}}Agrega archivos especificos\\ al Staging Area\end{tabular}                             & \textit{\textbf{git add \textless{}NombreArchivo1.extension\textgreater{}}}          & \textit{git add Codigo.c}                                                            \\ \hline
\begin{tabular}[c]{@{}l@{}}Agrega los archivos de un\\ mismo tipo al staging area\end{tabular}                    & \textit{\textbf{git add *\textless{}ExtensiónDeArchivo(.c, .h, .txt)\textgreater{}}} & \textit{git add *.c}                                                                 \\ \hline
\begin{tabular}[c]{@{}l@{}}Agrega todos los archivos \\ existentes en el workspace\\ al staging area\end{tabular} & \textit{\textbf{git add --all}}                                                      & \textit{git add --all}                                                               \\ \hline
\end{tabular}
\end{table}

Es importante destacar que, despues de que un archivo es añadido por primera vez al Staging Area, a partir de este momento el archivo será rastreado por el gestor de archivos (a menos que sea reseteado), aun cuando se le hagan cambios (sea modificado), sea preparado, o confirmado.

De igual manera, una vez que los archivos ya estan siendo rastreados por el gestor de archivos, este mismo comando es el que se utiliza para preparar archivos en el Workspace (pasandolos al staging area) cada que estos sean modificados. 

\textbf{\underline{Ejemplo:}}

A continuación se muestra un ejemplo de como agregar un archivo desde el workspace al staging area. Para que el ejemplo sea entendible, pondremos una comparación de cual es el estado del archivo antes y despues de ser agregado al Staging Area:

\begin{itemize}
    \item \textbf{\textit{Antes de agregar el archivo}}

    Antes de agregar el archivo, podemos ver que en el estado del git se encuentran tres archivos, los cuales estan en rojo, y se indica que ninguno ha sido asociado al gestor de archivos, por lo que ninguno se encuentra en el Staging Area, ni estan siendo rastreados.
    
        \begin{figure}[H]
            \centering
            \includegraphics[scale = 1]{Imagenes/8.3.2-1.png}
            \caption{Antes de agregar un archivo al staging area}
            \label{}
        \end{figure}
    
    \item \textbf{\textit{Ejecución del comando}}

    Procedemos a ejecutar el comando \textbf{\textit{git add}}, para agregar el archivo \textit{Codigo1.C} al staging area.
    
        \begin{figure}[H]
            \centering
            \includegraphics[scale = 1]{Imagenes/addFile.png}
            \caption{Agregando un archivo al staging area}
            \label{}
        \end{figure}
    
    \item \textbf{\textit{Despues de agregar el archivo}}

    Despues de agregar el archivo \textit{Codigo1.C} al staging area, podemos ver que este pasa a estar en color verde, indicando que ya esta preparado (y dentro del staging area). Del mismo modo este archivo es distiguido de los demas archivos, los cuales no han sido agregados, ni estan siendo rastreados.
    
        \begin{figure}[H]
            \centering
            \includegraphics[scale = 1]{Imagenes/8.3.2-3.png}
            \caption{Despues de agregar un archivo al staging area}
            \label{}
        \end{figure}
    
\end{itemize}

\subsection{Resetear un archivo del workspace}

Para que el gestor de archivos deje de rastrear un archivo y, por ende, deje de registrar los cambios que se le hagan al mismo, se debe resetear el archivo. Para resetear un archivo del workspace, se utiliza el comando \textit{\textbf{git reset}}, seguido del nombre del archivo y su extension.

\textbf{\underline{Ejemplo:}}

A continuación se muestra un ejemplo de como resetear un archivo del workspace. Para que el ejemplo sea entendible, pondremos una comparación del estado del archivo antes y despues de ser reseteado:

\begin{itemize}
    \item \textbf{\textit{Antes de resetear el archivo}}

    Antes de resetear un archivo, podemos ver que en el estado del git se encuentran tres archivos, los cuales ya estan preparados en el \emph{staging area}, y están siendo rastreados por el gestor de archivos.
    
        \begin{figure}[H]
            \centering
            \includegraphics[scale = 1]{Imagenes/8.3.3-1.png}
            \caption{Antes de resetear un archivo en el workspace}
            \label{}
        \end{figure}
    
    \item \textbf{\textit{Ejecución del comando}}

    Procedemos a ejecutar el comando \textbf{\textit{git reset}}, para resetear el archivo \textit{Codigo1.C} del workspace.
    
        \begin{figure}[H]
            \centering
            \includegraphics[scale = 1]{Imagenes/ResetFile.png}
            \caption{Reseteando un archivo del Workspace}
            \label{}
        \end{figure}
    
    \item \textbf{\textit{Despues de resetear el archivo}}

    Despues de resetear el archivo \textit{Codigo1.C}, podemos ver que este pasa a estar en color rojo, indicando que ya no está siendo rastreado, mientras que el resto de los archivos, continuan siendo rastreados por el gestor de archivos.
    
        \begin{figure}[H]
            \centering
            \includegraphics[scale = 1]{Imagenes/8.3.3-3.png}
            \caption{Despues de resetear un archivo del workspace}
            \label{}
        \end{figure}
    
\end{itemize}

\section{Manejo del repositorio local}

\subsection{Confirmar archivos}

Como se dijo con anterioridad, si bien una vez que los archivos se encuentran en el \textbf{\textit{staging area (área de preparación)}} ya estan siendo rastreados por el gestor de archivos, es necesario confirmarlos para poder considerarlos como una nueva versión registrada del proyecto almacenada de manera segura en la base de datos del gestor de archivos.

Para confirmar archivos se hace uso del comando \textbf{\textit{git commit – m}}, seguido de un comentario entre comillas. A continuación se muestra un ejemplo:

    \begin{figure}[H]
        \centering
        \includegraphics[scale = 1]{Imagenes/GitCommit.png}
        \caption{Confirmación de archivos de proyecto}
        \label{}
    \end{figure}

Es importante destacar que cada que se realiza una confirmación, se genera un \textbf{\textit{hash}}, que sirve como número de identificación único (ID) de cada confirmación. De igual manera, cuando se realiza una confirmación, git nos indica la rama en la que nos encontramos.

A continuación se muestra otro ejemplo de confirmación, remarcando cual es el \emph{hash} generado durante la confirmación, asi como la rama en la que nos encontramos:

\begin{figure}[H]
        \centering
        \includegraphics[scale = 1]{Imagenes/GitCommit2.png}
        \caption{Hash y Rama Actual}
        \label{}
    \end{figure}

\subsection{Ver registro de confirmaciones}

\section{Trabajando con el repositorio local}

\section{Creación y manejo del ramas}