\chapter{Despliegue y Configuración}
\label{chap:despliegue}

Este capítulo proporciona una guía completa para el despliegue del sistema de telemetría vehicular en diferentes entornos, desde desarrollo local hasta producción.

\section{Requisitos del Sistema}

\subsection{Hardware Mínimo}

\begin{itemize}[noitemsep]
    \item \textbf{CPU}: 2 cores a 2.0 GHz
    \item \textbf{RAM}: 4 GB mínimo, 8 GB recomendado
    \item \textbf{Almacenamiento}: 20 GB espacio libre
    \item \textbf{Red}: Conexión estable a internet
\end{itemize}

\subsection{Software Requerido}

\begin{itemize}[noitemsep]
    \item \textbf{Docker}: Versión 20.10 o superior
    \item \textbf{Docker Compose}: Versión 2.0 o superior
    \item \textbf{Git}: Para clonar el repositorio
    \item \textbf{Make}: Para comandos automatizados (opcional)
\end{itemize}

\section{Instalación Paso a Paso}

\subsection{Preparación del Entorno}

\begin{enumerate}[noitemsep]
    \item \textbf{Clonar el repositorio}:
    \begin{verbatim}
git clone <repository-url>
cd liese-av-geotel-mqtt-writter
    \end{verbatim}
    
    \item \textbf{Configurar variables de entorno}:
    \begin{verbatim}
cp .env.example .env
# Editar .env según las necesidades del entorno
    \end{verbatim}
    
    \item \textbf{Verificar instalación de Docker}:
    \begin{verbatim}
docker --version
docker-compose --version
    \end{verbatim}
\end{enumerate}

\subsection{Despliegue Automático}

Para una instalación rápida y completa:

\begin{verbatim}
# Instalación completa con un comando
make setup

# O manualmente:
docker-compose build
docker-compose up -d
make create-tables
make sample-data
\end{verbatim}

\section{Configuración de Variables de Entorno}

\subsection{Archivo .env}

El archivo \texttt{.env} contiene todas las configuraciones del sistema:

\begin{verbatim}
# Base de Datos
POSTGRES_DB=geotel_db
POSTGRES_USER=geotel_user
POSTGRES_PASSWORD=geotel_password
POSTGRES_HOST=postgres
POSTGRES_PORT=5432

# MQTT Broker
MQTT_BROKER=host.docker.internal
MQTT_PORT=1883
MQTT_USERNAME=
MQTT_PASSWORD=

# pgAdmin
PGADMIN_DEFAULT_EMAIL=admin@geotel.com
PGADMIN_DEFAULT_PASSWORD=admin123

# Logging
LOG_LEVEL=INFO
LOG_FILE=/app/logs/mqtt_writer.log
\end{verbatim}

\subsection{Configuraciones por Entorno}

\subsubsection{Desarrollo Local}
\begin{verbatim}
MQTT_BROKER=localhost
LOG_LEVEL=DEBUG
POSTGRES_PASSWORD=dev_password
\end{verbatim}

\subsubsection{Producción}
\begin{verbatim}
MQTT_BROKER=production-mqtt-server.com
LOG_LEVEL=ERROR
POSTGRES_PASSWORD=secure_production_password
MQTT_USERNAME=prod_user
MQTT_PASSWORD=secure_mqtt_password
\end{verbatim}

\section{Comandos de Gestión}

\subsection{Tabla de Comandos Makefile}

\begin{table}[H]
\centering
\small
\begin{tabular}{|l|p{8cm}|}
\hline
\textbf{Comando} & \textbf{Descripción} \\
\hline
\texttt{make setup} & Configuración completa inicial del sistema \\
\hline
\texttt{make build} & Construir todas las imágenes Docker \\
\hline
\texttt{make up} & Levantar todos los servicios \\
\hline
\texttt{make down} & Detener todos los servicios \\
\hline
\texttt{make logs} & Ver logs del servicio principal \\
\hline
\texttt{make logs-all} & Ver logs de todos los servicios \\
\hline
\texttt{make status} & Ver estado de todos los contenedores \\
\hline
\texttt{make restart} & Reiniciar el servicio principal \\
\hline
\texttt{make clean} & Limpiar contenedores, volúmenes e imágenes \\
\hline
\texttt{make rebuild} & Reconstruir todo desde cero \\
\hline
\texttt{make create-tables} & Crear esquema de base de datos \\
\hline
\texttt{make sample-data} & Insertar datos de ejemplo \\
\hline
\texttt{make simulate} & Ejecutar simulador de telemetría \\
\hline
\texttt{make test-mqtt} & Enviar mensajes MQTT de prueba \\
\hline
\texttt{make db-shell} & Acceder a shell de PostgreSQL \\
\hline
\texttt{make check-health} & Verificar salud de servicios \\
\hline
\texttt{make debug} & Mostrar información de diagnóstico \\
\hline
\end{tabular}
\end{table}

\section{Configuración de Red}

\subsection{Puertos Utilizados}

\begin{table}[H]
\centering
\begin{tabular}{|l|l|l|}
\hline
\textbf{Servicio} & \textbf{Puerto} & \textbf{Descripción} \\
\hline
PostgreSQL & 5432 & Base de datos principal \\
\hline
MQTT Broker & 1883 & Comunicación MQTT \\
\hline
pgAdmin & 8080 & Interfaz web de administración \\
\hline
MQTT WebSocket & 9001 & MQTT sobre WebSocket (opcional) \\
\hline
\end{tabular}
\end{table}

\section{Monitoreo y Logging}

\subsection{Configuración de Logs}

Los logs se configuran en el archivo de configuración Python:

\begin{verbatim}
import logging

logging.basicConfig(
    level=logging.INFO,
    format='%(asctime)s - %(name)s - %(levelname)s - %(message)s',
    handlers=[
        logging.FileHandler('/app/logs/mqtt_writer.log'),
        logging.StreamHandler()
    ]
)
\end{verbatim}

\section{Seguridad}

\subsection{Configuraciones de Seguridad Básica}

\begin{enumerate}[noitemsep]
    \item \textbf{Cambiar credenciales por defecto}:
    \begin{verbatim}
# En .env de producción
POSTGRES_PASSWORD=secure_random_password_123
PGADMIN_DEFAULT_PASSWORD=another_secure_password_456
    \end{verbatim}
    
    \item \textbf{Configurar autenticación MQTT}:
    \begin{verbatim}
MQTT_USERNAME=mqtt_user
MQTT_PASSWORD=mqtt_secure_password
    \end{verbatim}
    
    \item \textbf{Restringir acceso a puertos}:
    Solo exponer puertos necesarios externamente
\end{enumerate}

\section{Backup y Recuperación}

\subsection{Backup de Base de Datos}

Script automatizado para backup:

\begin{verbatim}
#!/bin/bash
# backup_db.sh

DATE=$(date +%Y%m%d_%H%M%S)
BACKUP_DIR="/backups"
DB_NAME="geotel_db"

docker-compose exec postgres pg_dump -U geotel_user $DB_NAME > \
  $BACKUP_DIR/backup_${DB_NAME}_${DATE}.sql

# Comprimir backup
gzip $BACKUP_DIR/backup_${DB_NAME}_${DATE}.sql

# Limpiar backups antiguos (más de 30 días)
find $BACKUP_DIR -name "backup_*.sql.gz" -mtime +30 -delete
\end{verbatim}

\subsection{Restauración}

Para restaurar desde backup:

\begin{verbatim}
# Detener servicios
make down

# Restaurar base de datos
zcat /backups/backup_geotel_db_20241201_120000.sql.gz | \
docker-compose exec -T postgres psql -U geotel_user geotel_db

# Reiniciar servicios
make up
\end{verbatim}
