\chapter{Mejoras Futuras}
\label{ch:mejoras_futuras}

El sitio web de LIESE es una plataforma robusta y funcional, pero siempre hay espacio para la mejora y la expansión. A continuación, se presenta una lista de posibles mejoras que podrían implementarse en el futuro para enriquecer la funcionalidad, la seguridad y la experiencia de usuario del sitio.

\begin{itemize}
    \item \textbf{Certificación SSL:} Implementar un certificado SSL/TLS (por ejemplo, a través de Let's Encrypt) para asegurar que toda la comunicación entre el cliente y el servidor esté cifrada (HTTPS).

    \item \textbf{Mejoras de Ciberseguridad:} Realizar una auditoría de seguridad exhaustiva y aplicar parches y configuraciones para fortalecer el sitio contra vulnerabilidades comunes.

    \item \textbf{Botón de Patrocinio:} Integrar una pasarela de pago como Stripe o PayPal para facilitar las donaciones y patrocinios al laboratorio.

    \item \textbf{CI/CD (Integración y Despliegue Continuos):} Configurar un pipeline de CI/CD utilizando herramientas como GitHub Actions o Travis CI para automatizar las pruebas y el despliegue de nuevas versiones del sitio.

    \item \textbf{Soporte Multiidioma:} Añadir soporte para múltiples idiomas (como el inglés) para ampliar el alcance del sitio a una audiencia internacional.

    \item \textbf{Dashboard Administrativo:} Desarrollar un panel de administración personalizado y más visual para la gestión de contenido, más allá del panel por defecto de Django.

    \item \textbf{Sistema de Búsqueda:} Implementar un motor de búsqueda más avanzado (por ejemplo, con Elasticsearch o Algolia) que permita a los usuarios encontrar rápidamente proyectos, artículos o noticias.

    \item \textbf{Integración con Google Calendar:} Sincronizar los eventos del laboratorio con un calendario de Google público para que los usuarios puedan suscribirse y recibir notificaciones.

    \item \textbf{Modo Claro/Oscuro:} Ofrecer a los usuarios la opción de cambiar entre un tema claro y uno oscuro para mejorar la accesibilidad y la comodidad visual.

    \item \textbf{Sistema de Comentarios:} Añadir una sección de comentarios en los artículos y noticias para fomentar la interacción y el debate.

    \item \textbf{Pruebas Unitarias y de Integración:} Desarrollar un conjunto completo de pruebas automatizadas para garantizar la estabilidad y la calidad del código en futuros desarrollos.

    \item \textbf{Dockerización:} Empaquetar la aplicación y sus dependencias en contenedores de Docker para simplificar el desarrollo, el despliegue y la escalabilidad.

    \item \textbf{Análisis y Monitoreo:} Integrar herramientas como Google Analytics para obtener métricas de uso del sitio y Sentry para el monitoreo de errores en tiempo real.
\end{itemize}