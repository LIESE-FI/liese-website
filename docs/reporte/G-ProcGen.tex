\chapter{Primeros pasos con GitHub}

\section{Creación de repositorio en línea}

En esta sección se mostrará el proceso para la creación de un repositorio en línea.

\subsubsection{1. En la página principal de GitHub aparece la opción marcada en verde para crear un nuevo repositorio.}

Se tiene que estar en la página principal \href{https://github.com/}{https://github.com/} una vez que previamente se ha iniciado sesión en la cuenta de GitHub para ver esta opción.

\begin{figure}[H]
    \centering
    \includegraphics[width=0.8\textwidth]{Imagenes/Screenshot_13.png}
    \caption{Opción de crear un primer repositorio}
    \label{fig:onlinerepo1}
\end{figure}

\begin{figure}[H]
    \centering
    \includegraphics[width=0.8\textwidth]{Imagenes/Screenshot_13_2.png}
    \caption{Con repositorios previos}
    \label{fig:onlinerepo1}
\end{figure}

\subsubsection{2. Asignamos el nombre del repositorio*, una breve descripción y definimos el tipo de acceso.}

*El nombre del repositorio debe de usar solo caracteres ASCII, sin acentos, sin espacios y sin empezar por números.

\begin{figure}[H]
    \centering
    \includegraphics[width=0.8\textwidth]{Imagenes/Screenshot_14.png}
    \caption{Creación de repositorio}
    \label{fig:onlinerepo1}
\end{figure}

\subsubsection{3. Agregamos un archivo README y creamos el repositorio.}

\begin{figure}[H]
    \centering
    \includegraphics[width=0.8\textwidth]{Imagenes/Screenshot_15.png}
    \caption{Creación de repositorio repositorio}
    \label{fig:onlinerepo1}
\end{figure}

\subsubsection{4. Se ha creado el repositorio y se muestra el inicio.}

\begin{figure}[H]
    \centering
    \includegraphics[width=0.8\textwidth]{Imagenes/Screenshot_16.png}
    \caption{Repositorio creado}
    \label{fig:onlinerepo1}
\end{figure}

\section{Conectar una carpeta de proyecto con un repositorio en GitHub}

En esta sección se muestra el proceso para subir un proyecto contenido en una carpeta a un repositorio vacío en GitHub.

\noindent \textbf{1.} Posicionarse dentro de la carpeta del proyecto usando el explorador de archivos. Emplear alguna carpeta de proyecto que contenga archivos.

\begin{figure}[H]
    \centering
    \includegraphics[width=0.6\textwidth]{Imagenes/Screenshot_17.png}
    \caption{Contenido de la carpeta de proyecto}
    \label{fig:onlinerepo1}
\end{figure}

\noindent \textbf{2.} Haciendo click derecho seleccionar la opción de abrir la consola de git bash.

\begin{figure}[H]
    \centering
    \includegraphics[width=0.6\textwidth]{Imagenes/Screenshot_18.png}
    \caption{Consola de git bash}
    \label{fig:onlinerepo1}
\end{figure}

\noindent \textbf{3.} En la consola de git bash se inicializa el repositorio con el comando ''git init" . Con esto el repositorio local se ha inicializado y se da seguimiento al contenido de la carpeta. Se mostrará un mensaje de que se ha inicializado un repositorio de git vacío. Posteriormente podemos ejecutar un comando ''git status", el cual nos mostrará la rama actual de trabajo y los cambios detectados marcados en rojo.

\begin{figure}[H]
    \centering
    \includegraphics[width=0.6\textwidth]{Imagenes/Screenshot_19.png}
    \caption{Inicialización del repositorio}
    \label{fig:onlinerepo1}
\end{figure}

\noindent \textbf{4.} Una vez que se ha inicializado el repositorio podemos añadir los archivos del proyecto. A modo de ejemplo se añaden todos los archivos marcados en rojo con el comando ''git add .'' con lo cual todos los archivos pasarán al \textit{staging area}. Esto se puede verificar con la ejecución posterior del comando ''git status'' con lo cual se mostrarán en verde los archivos añadidos.

\begin{figure}[H]
    \centering
    \includegraphics[width=0.6\textwidth]{Imagenes/Screenshot_20.png}
    \caption{Archivos en \textit{staging area}}
    \label{fig:onlinerepo1}
\end{figure}

\noindent \textbf{5.} En esta etapa se usará el comando ''git commit -m '' seguido de un comentario entre comillas para crear un punto de guardado del repositorio en su estado actual. Se añade un comentario que permita identificar de manera breve y general el motivo del guardado del estado actual. Con ''git status'' se puede verificar que no quedan cambios de la etapa anterior.

\begin{figure}[H]
    \centering
    \includegraphics[width=0.6\textwidth]{Imagenes/Screenshot_21.png}
    \caption{Archivos}
    \label{fig:onlinerepo1}
\end{figure}

\noindent \textbf{6.} Se asocia un repositorio remoto con el repositorio local, para esto se usa el comando ''git add remote'' seguido de dos argumentos, un nombre remoto, que generalmente es origin y el link del repositorio remoto.

\begin{figure}[H]
    \centering
    \includegraphics[width=0.6\textwidth]{Imagenes/Screenshot_22.png}
    \caption{Añadir un repositorio remoto}
    \label{fig:onlinerepo1}
\end{figure}

El link del repositorio remoto se consigue a través de la opción de ''Code'' y posteriormente en la pestaña de HTTPS encontramos el link del repositorio que se necesita conectar y que está vacío.

\begin{figure}[H]
    \centering
    \includegraphics[width=0.9\textwidth]{Imagenes/Screenshot_23.png}
    \caption{Link de repositorio remoto}
    \label{fig:onlinerepo1}
\end{figure}

\noindent \textbf{7.} Para subir los cambios al repositorio en línea se usa el comando ''git push'' seguido de dos argumentos, que son el nombre remoto y el nombre de la rama desde la cual se tienen los cambios.

\begin{figure}[H]
    \centering
    \includegraphics[width=0.7\textwidth]{Imagenes/Screenshot_24.png}
    \caption{Cambios subidos a repositorio remoto}
    \label{fig:onlinerepo1}
\end{figure}

\begin{figure}[H]
    \centering
    \includegraphics[width=0.7\textwidth]{Imagenes/Screenshot_25.png}
    \caption{Cambios subidos a repositorio remoto}
    \label{fig:onlinerepo1}
\end{figure}

