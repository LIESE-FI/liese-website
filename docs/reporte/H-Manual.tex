\chapter{Manual de Usuario}
\label{chap:manual}

Este capítulo proporciona una guía completa para usuarios finales del sistema de telemetría vehicular, incluyendo interfaces de administración, consultas comunes y resolución de problemas.

\section{Acceso a las Interfaces del Sistema}

\subsection{pgAdmin - Administración de Base de Datos}

pgAdmin es la interfaz web principal para administrar la base de datos PostgreSQL.

\subsubsection{Acceso Inicial}
\begin{enumerate}[noitemsep]
    \item Abrir navegador web en: \texttt{http://localhost:8080}
    \item Credenciales por defecto:
    \begin{itemize}
        \item Email: \texttt{admin@geotel.com}
        \item Password: \texttt{admin123}
    \end{itemize}
    \item Configurar conexión al servidor PostgreSQL
\end{enumerate}

\subsubsection{Configuración del Servidor}
Una vez en pgAdmin, agregar el servidor de base de datos:

\begin{table}[H]
\centering
\begin{tabular}{|l|l|}
\hline
\textbf{Campo} & \textbf{Valor} \\
\hline
Nombre & Geotel PostgreSQL \\
\hline
Host & postgres \\
\hline
Puerto & 5432 \\
\hline
Base de datos & geotel\_db \\
\hline
Usuario & geotel\_user \\
\hline
Contraseña & geotel\_password \\
\hline
\end{tabular}
\end{table}

\section{Consultas Comunes}

\subsection{Visualización de Datos de Telemetría}

\subsubsection{Últimos Registros por Unidad}
\begin{verbatim}
SELECT 
    u.name AS unidad,
    t.parameter AS parametro,
    t.value AS valor,
    t.timestamp AS fecha_hora
FROM "Telemetries" t
JOIN "Units" u ON t.unit_id = u.id
WHERE t.timestamp > NOW() - INTERVAL '1 hour'
ORDER BY t.timestamp DESC
LIMIT 20;
\end{verbatim}

\subsubsection{Estadísticas de Combustible}
\begin{verbatim}
SELECT 
    u.name AS unidad,
    ROUND(AVG(t.value), 2) AS promedio_combustible,
    ROUND(MIN(t.value), 2) AS minimo,
    ROUND(MAX(t.value), 2) AS maximo,
    COUNT(*) AS total_registros
FROM "Telemetries" t
JOIN "Units" u ON t.unit_id = u.id
WHERE t.parameter = 'Combustible'
  AND t.timestamp > NOW() - INTERVAL '24 hours'
GROUP BY u.id, u.name
ORDER BY promedio_combustible DESC;
\end{verbatim}

\section{Uso del Simulador MQTT}

\subsection{Simulación Básica}

Para generar datos de prueba:

\begin{verbatim}
# Simular una unidad específica
make simulate-unit UNIT=1

# Simular con delay personalizado
docker-compose exec mqtt-writer python simulate_mqtt.py --unit 2 --delay 5

# Simular múltiples unidades
for i in {1..3}; do
    make simulate-unit UNIT=$i &
done
\end{verbatim}

\subsection{Parámetros del Simulador}

\begin{table}[H]
\centering
\begin{tabular}{|l|l|p{6cm}|}
\hline
\textbf{Parámetro} & \textbf{Rango} & \textbf{Descripción} \\
\hline
Combustible & 0 - 100 & Porcentaje de combustible en el tanque \\
\hline
Velocidad & 0 - 120 & Velocidad en kilómetros por hora \\
\hline
RPM & 600 - 4000 & Revoluciones por minuto del motor \\
\hline
Temperatura & 70 - 110 & Temperatura del motor en grados Celsius \\
\hline
Panic & 0 - 1 & Estado del botón de pánico (0=normal, 1=activado) \\
\hline
\end{tabular}
\end{table}

\section{Monitoreo del Sistema}

\subsection{Verificación del Estado}

\subsubsection{Estado de Servicios}
\begin{verbatim}
# Ver estado de contenedores
make status

# Verificar salud de servicios
make check-health

# Ver logs en tiempo real
make logs
\end{verbatim}

\section{Envío Manual de Datos MQTT}

\subsection{Comandos Básicos}

Para enviar datos manualmente al sistema:

\begin{verbatim}
# Datos de combustible
mosquitto_pub -h localhost -t "U1_Combustible" -m "75.5"

# Datos de velocidad
mosquitto_pub -h localhost -t "U2_Velocidad" -m "95.2"

# Datos de temperatura
mosquitto_pub -h localhost -t "U1_Temperatura" -m "87"

# Botón de pánico
mosquitto_pub -h localhost -t "U3_Panic" -m "1"
\end{verbatim}

\section{Resolución de Problemas Comunes}

\subsection{No se Reciben Mensajes MQTT}

\begin{enumerate}[noitemsep]
    \item Verificar estado del broker MQTT:
    \begin{verbatim}
mosquitto_pub -h localhost -t test -m "hello"
    \end{verbatim}
    
    \item Revisar logs del sistema:
    \begin{verbatim}
make logs
    \end{verbatim}
    
    \item Verificar formato del topic:
    Debe ser \texttt{U\{número\}\_\{parámetro\}}
\end{enumerate}

\subsection{Datos No Aparecen en la Base}

\begin{enumerate}[noitemsep]
    \item Verificar que la unidad exista:
    \begin{verbatim}
SELECT * FROM "Units" WHERE id = 1;
    \end{verbatim}
    
    \item Revisar logs de errores:
    \begin{verbatim}
make debug
    \end{verbatim}
    
    \item Probar inserción manual:
    \begin{verbatim}
make test-mqtt
    \end{verbatim}
\end{enumerate}
