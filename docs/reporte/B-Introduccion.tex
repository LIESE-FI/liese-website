\chapter{Introducción}

\section{Presentación del LIESE}
El Laboratorio de Instrumentación Electrónica de Sistemas Espaciales (LIESE) es una unidad académica adscrita a la División de Ingeniería Eléctrica de la Facultad de Ingeniería de la Universidad Nacional Autónoma de México (UNAM). Su misión es impulsar la formación de especialistas y el desarrollo de tecnología en el área espacial, con énfasis en el diseño, construcción e implementación de sistemas electrónicos para aplicaciones satelitales, instrumentación embebida, Internet de las Cosas (IoT) e inteligencia artificial. El LIESE promueve la colaboración interdisciplinaria y la innovación tecnológica, contribuyendo al avance científico y al bienestar social en México.

\section{Objetivos del sitio web}
El sitio web del LIESE tiene como objetivos principales:
\begin{itemize}
    \item Difundir los proyectos, logros y actividades del laboratorio a la comunidad académica y al público en general.
    \item Facilitar el reclutamiento de nuevos talentos y la vinculación con otras instituciones.
    \item Proveer un canal de comunicación actualizado sobre eventos, oportunidades y noticias relevantes.
    \item Servir como plataforma de gestión y administración de contenido para los miembros del laboratorio.
\end{itemize}

\section{Alcance del sistema web}
El sistema web desarrollado abarca las siguientes funcionalidades:
\begin{itemize}
    \item Publicación y gestión de artículos, tesis y noticias.
    \item Visualización de proyectos y actividades del laboratorio.
    \item Calendario de eventos académicos y de divulgación.
    \item Sección de líderes de proyecto y miembros del laboratorio.
    \item Sistema de oportunidades con autenticación de dos factores (2FA) para solicitudes.
    \item Panel de administración basado en Django para la gestión de contenido.
    \item Interfaz responsiva y moderna, accesible desde dispositivos móviles y de escritorio.
\end{itemize}

\section{Justificación del desarrollo}
El desarrollo de un sitio web institucional es fundamental para fortalecer la presencia digital del LIESE, facilitar la comunicación interna y externa, y promover la transparencia y el acceso a la información. La plataforma permite centralizar la gestión de contenido, automatizar procesos administrativos y ofrecer una experiencia de usuario moderna y segura. Además, la integración de mecanismos de seguridad como la verificación por correo electrónico y la autenticación de dos factores contribuye a la protección de los datos y la confianza de los usuarios.
